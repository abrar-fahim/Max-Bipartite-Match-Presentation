\documentclass[10pt]{beamer}
\usetheme{Berkeley}
\usecolortheme{beaver}
\usepackage{helvet}
\usepackage{tikz}
%\usefonttheme{structuresmallcapsserif}

\begin{document}

%info in title page
\title{Maximum Bipartite Matching}
\author{Nafiz Imtiaz, Abrar Fahim, S. S. Somik}
\institute{Department of CSE, BUET, Dhaka 1000, Bangladesh}
\date{\today}
\subtitle{CSE 300 Presentation}

\begin{frame}
    \frametitle{title}
    
	First 2 slides contain sample blocks, columns and stuff
    
    \begin{block}{Remark}
    in block
    \end{block}
    
    \begin{alertblock}{important theorem}
    in alertblock
    \end{alertblock}
    
    \begin{examples} 
    in users block
    \end{examples}
    
    text in frame
    
    \alert{this is alert}


\end{frame}

\begin{frame}
	\frametitle{columns}
	\begin{columns}
		\column{0.5\textwidth}
		text in first column
		 $$E = mc^2$$
		\begin{itemize}
			
			\item first item
			\item hello there
		\end{itemize}
		
		\column{0.5\textwidth}
		text in second column
	\end{columns}
\end{frame}

\begin{frame}
	\frametitle{Real World Applications}
\end{frame}

\frame{\titlepage}

\begin{frame}
	\frametitle{Bipartite Graph Definition}
	
	\begin{tikzpicture}
		\draw (1,2) -- (2,4);
		\node{1};
	\end{tikzpicture}
\end{frame}

\begin{frame}
	\frametitle{Maximum Matching}
\end{frame}

\begin{frame}
	\frametitle{Max Flow Problem Overview}
\end{frame}

\begin{frame}
	\frametitle{Relating Max Flow Problem with Maximum Matching}
\end{frame}


\begin{frame}
	\frametitle{Max Flow problem Algorithm}
	\begin{itemize}
		\item solving max flow using ford fulkerson takes O(ef*) time, where f* = max flow, but in our problem, f* = n = no of vertices
		and no of edges = m + 2n, m = no of edges in bipartite graph and 2n edges are added later to convert the problem into max flow problem. So running time$$ = O((m + 2n)n) = O(mn + n^2) = O(mn) $$
		
		
		\item but, we can use a simpler way to solve this since edges are unweighted, although we get same time complexity
		
		\item even simpler way is a recursive way which is shown as "Alternative approach" in the fancy slides from the internet

	\end{itemize}
	
		
\end{frame}



\begin{frame}
	\frametitle{Solving Main Problem}
\end{frame}

\begin{frame}
	\frametitle{Time Complexity}
\end{frame}

\begin{frame}
	\frametitle{Alternate solutions}
\end{frame}

\begin{frame}
	\frametitle{References}
\end{frame}
	
	
\end{document}